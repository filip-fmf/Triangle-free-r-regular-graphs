\documentclass[a4paper,12pt]{article}
\usepackage[utf8]{inputenc}
\usepackage[T1]{fontenc}
\usepackage[slovene]{babel}
\usepackage{lmodern}  
\usepackage{amsmath,amssymb}
\usepackage{booktabs}
\usepackage{graphicx}
\usepackage{float}

\newcommand{\fn}[1]{\texttt{#1}}

\begin{document}

\begin{titlepage}
    \centering

    {\Large Univerza v Ljubljani\\
    Fakulteta za matematiko in fiziko\par}
    
    \vspace{3cm}
    
    {\Huge \textbf{r-regularni grafi brez trikotnikov}\par}
    
    \vspace{3cm}
    
    {\large Avtorja:\par}
    \vspace{0.2cm}
    {\large Luka Hodnik\\
    Filip Cerkovnik\par}
    
    \vfill
    
    {\large Februar 2026\par}
\end{titlepage}

\tableofcontents
\newpage

\section{Uvod}
Naj bo $r$ liho število. V projektu obravnavamo problem karakterizacije
povezanih, trikotnikov prostih in $r$-regularnih grafov, ki zadoščajo
naslednjim pogojem:
\begin{enumerate}
    \item Za vsako vozlišče $v$ je množica $N(v)$ maksimalna močna liha
    neodvisna množica.
    \item Obstaja vsaj eno vozlišče $v$, za katero je množica $N(v)$
    maksimalna močna liha neodvisna množica.
    \item Velja $\alpha_{od}(G) = r$.
\end{enumerate}

Gre za tri različne probleme, pri čemer velja zaporedje implikacij
\[
(i)\Rightarrow(ii)\Rightarrow(iii).
\]

Znana sta dva primera grafov, ki zadoščata pogoju (i), in sicer
Petersenov graf ter polni dvodelni graf $K_{r,r}$, kjer je $r$ liho.
Pomagali si bomo tudi z 2 znanima dejstvoma iz literature:
\begin{enumerate}
    \item Nujen pogoj za izpolnjevanje pogojev $(i)$ in $(ii)$ je, da ima graf
    premer največ $3$.
    \item Kar zadeva pogoj $(iii)$, pri katerem $r \ge 3$ ni nujno liho število,
    zaradi odsotnosti pogojev na sosedstva $N(v)$ velja zgornja meja
    za red grafa
    \[
    |V(G)| \le r(r^2 - 1).
    \]
\end{enumerate}


\section{Cilji naloge}
Cilji naloge so trije: 
\begin{enumerate}
    \item Poiskati grafe, ki ustrezajo pogojem $(i)$, $(ii)$ in $(iii)$ za različne vrednosti $n$ in $r$.
    \item Uporabiti naključne sprehode med večjimi grafi s premerom največ 3.
    \item Preveriti pogoje na močno regularnih grafih $\mathrm{srg}(n,r,\lambda = 0,\mu)$.
\end{enumerate}

\section{Iskanje grafov, ki zadoščajo pogojem $(i)$, $(ii)$ in $(iii)$}

\subsection{$(i)$ del}
V tem delu obravnavamo grafe, za katere velja, da ima vsako vozlišče
maksimalno močno liho neodvisno množico enake velikosti.

\subsubsection*{Opis pristopa}

Ker gre za zelo restriktiven pogoj, smo se omejili na sistematično iskanje
grafov majhnega reda. Pri tem smo upoštevali znano dejstvo, da je nujen
pogoj za izpolnitev tega pogoja, da ima graf premer največ $3$.
S tem smo bistveno zmanjšali iskalni prostor.

Preverjanje lastnosti smo izvedli z računalniškim programom, ki za vsak
kandidatni graf izračuna velikost maksimalne močne lihe neodvisne množice
glede na posamezno vozlišče.

\subsubsection*{Ugotovitve}

Rezultati kažejo, da je pogoj izjemno močan in ga izpolnjuje zelo malo grafov.
Pri majhnih redih se pojavi le nekaj izjem, ki so večinoma dobro znani
in imajo visoko stopnjo simetrije.

Med najbolj značilnimi primeri se pojavita Petersenov graf in družina grafov
$K_{r,r}$, kjer je $r$ liho število. Gre za grafe z majhnim premerom in
enakomerno razporejeno strukturo. To nakazuje, da je visoka simetrija grafa ključnega pomena za izpolnjevanje pogoja iz tega dela naloge. 

Podrobni primeri in slike so zbrani v prilogi.

\subsection{$(ii)$ del}
V tem delu obravnavamo grafe, za katere velja, da ima neko vozlišče
maksimalno močno liho neodvisno množico enake velikosti.

\subsubsection*{Opis pristopa}

Ta pogoj je strukturno manj zahteven kot v $(i)$ delu, saj ne zahteva, da so vse
maksimalne močne lihe neodvisne množice enake velikosti, temveč le, da zanje
obstaja reprezentant med maksimalnimi množicami, povezanimi s posameznimi
vozlišči.

Pristop temelji na generiranju in analizi grafov majhnega reda. Za vsak graf
preverimo, ali za vsako vozlišče $v$ obstaja taka maksimalna močno liha
neodvisna množica $N(v)$, ki je hkrati tudi maksimalna močno liha neodvisna
množica za neko vozlišče $u$ v grafu (kjer $u$ je lahko tudi $v$).
To pomeni, da vsaka lokalno maksimalna množica (izbrana iz perspektive
nekaterega vozlišča) doseže globalno maksimalno velikost glede na neko vozlišče.

\subsubsection*{Ugotovitve}

Rezultati kažejo, da je ta pogoj precej bolj pogost kot v $(i)$ delu.
Poleg visoko simetričnih grafov (kot so Petersenov graf in popolni dvodelni
grafi $K_{r,r}$ z lihim $r$ pogoj izpolnjuje tudi več drugih grafov kot pri $(i)$

Pogoj je torej veliko bolj vključujoč in se pojavlja v raznolikih strukturah,
ki dopuščajo, da se z vidika vsakega vozlišča da identificirati neko
maksimalno močno liho neodvisno množico, ki je enake velikosti kot neka
globalno maksimalna množica.

Na koncu je nekaj grafov, ki ustrezajo temu pogoju.
\subsection{$(iii)$) del}

V tem delu obravnavamo najšibkejši pogoj, in sicer da velja
\[
\alpha_{od}(G) = r.
\]
Pri tem ne zahtevamo posebnih pogojev glede posameznih vozlišč.
Ker je pogoj bistveno šibkejši kot v prejšnjih delih sva dobila veliko več grafov, ki ustreza temu pogoju.
Izkazalo se je, da ta pogoj dopušča bistveno večjo raznolikost grafov.
Čeprav mora graf še vedno izpolnjevati osnovne strukturne omejitve,
kot je majhen premer, se pojavijo številni primeri z zelo različno lokalno
strukturo.
Rezultati potrjujejo, da pogoj $\alpha_{od}(G)=r$ sam po sebi ne vsiljuje
visoke simetrije grafa, temveč predvsem omejuje paritetne lastnosti
sosedstev. Spodaj lahko vidimo tabele koliko grafov ustreza temu pogoju pri različnem parametru $n$ in parametru $r$. Rezultati niso napisani za vse možne vrednosti $n$, saj pri prevelikih vrednostih program predolgo deluje in nama zmanjka RAMA. Podrobnejši primeri so zbranni v prilogi.


\begin{table}[H]
    \centering
    {\bfseries $r = 3$}\\[0.5em]
    \begin{tabular}{cr}
        \toprule
        $n$ & Število grafov z $\alpha_{\mathrm{od}}(G) = r$ \\
        \midrule
        4 & 0 \\
        6 & 1 \\
        8 & 0 \\
        10 & 2 \\
        12 & 11 \\
        14 & 23 \\
        16 & 13 \\
        18 & 0 \\
        20 & 0 \\
        \bottomrule
    \end{tabular}
    \caption{Število grafov z $\alpha_{\mathrm{od}}(G)=r$, pri različnih $n$ za $r=3$}
\end{table}

\begin{table}[H]
    \centering
    {\bfseries $r = 5$}\\[0.5em]
    \begin{tabular}{cr}
        \toprule
        $n$ & Število grafov z $\alpha_{\mathrm{od}}(G) = r$ \\
        \midrule
        6 & 0 \\
        8 & 0 \\
        10 & 1 \\
        12 & 0 \\
        14 & 0\\
        16 & 32 \\
        18 & >30.000 \\
        \bottomrule
    \end{tabular}
    \caption{Število grafov z $\alpha_{\mathrm{od}}(G)=r$, pri različnih $n$ za $r=5$}
\end{table}

\begin{table}[H]
    \centering
    {\bfseries $r = 7$}\\[0.5em]
    \begin{tabular}{cr}
        \toprule
        $n$ & Število grafov z $\alpha_{\mathrm{od}}(G) = r$ \\
        \midrule
        8 & 0 \\
        10 & 0 \\
        12 & 0 \\
        14 & 1\\
        16 & 0 \\
        18 & 0 \\
        \bottomrule
    \end{tabular}
    \caption{Število grafov z $\alpha_{\mathrm{od}}(G)=r$, pri različnih $n$ za $r=7$}
\end{table}

\section{Naključni sprehodi med večjimi grafi s premerom največ 3}

Za eksperimentalno analizo smo izvedli naključne sprehode po prostoru $r$-regularnih grafov brez trikotnikov.
Spremljali smo število vozlišč, katerih soseska tvori maksimalen \emph{strong odd independent set} (SOIS). 
Začetni graf je bil izbran kot znan primer (Petersenov graf ali $K_{r,r}$ za lihe $r$), če je bil na voljo;
sicer smo uporabili naključni $r$-regularen graf brez trikotnikov.

Med sprehodi z naključnimi zamenjavami povezav (angl. \emph{random 2-switch}) smo beležili dva parametra: premer grafa in število vozlišč, katerih soseska je maksimalen SOIS.
Rezultati kažejo, da so grafi s premerom $\leq 3$, pri katerih vsaj ena soseska tvori maksimalen SOIS, izjemno redki.
Pri velikih grafih se skoraj nobena soseska ne obnaša kot maksimalni SOIS. 
Vizualizacija števila takšnih sosesk v odvisnosti od koraka sprehoda potrjuje, da pogoja (i) in (ii) postaneta zelo restriktivna pri večjih grafih,
kar je skladno s teoretičnimi pričakovanji.

\section{Preizkušanje močno regularnih grafov tipa $\mathrm{srg}(n,r,\lambda = 0,\mu)$}

Opazimo, da znana primera grafov (Petersenov graf in graf $K_{r,r}$),
ki zadoščata pogoju iz dela $(i)$, spadata v razred močno regularnih grafov.
To naju je spodbudilo, da sva preverjanje pogojev $(i)$-$(iii)$ preizkusila tudi na drugih močno regularnih grafih tipa $\mathrm{srg}(n,r,\lambda = 0,\mu)$.

Preverjanje sva najprej preizkusila na Petersenovem grafu, ki ima vrednosti
$(10,3,0,1)$, in Clebschovem grafu, ki je še en znan primer strogo
regularnega grafa z vrednostmi $(16,5,0,2)$.
Rezultat je zelo pričakovan: Petersenov graf zadošča vsem pogojem $(i)$-$(iii)$, Clebschov graf pa nobenemu.

Preverjanje sva nato še preizkusila na grafih $K_{r,r}$ za $r=3,5,7$ in
pričakovano vsi zadoščajo vsem pogojem $(i)$-$(iii)$.

Nato sva iskanje preizkusila še na naključnih strogo regularnih grafih. Za ilustracijo razlik med posameznimi pogoji sva izbrali po en
reprezentativen primer grafa za vsako izmed kombinacij
$(i),(ii),(iii) = (\text{True},\text{True},\text{True})$,
$(\text{False},\text{True},\text{True})$ in
$(\text{False},\text{False},\text{True})$.
Primeri so bili izbrani iz nabora naključno generiranih grafov.

\begin{figure}[H]
\centering
\includegraphics[width=0.32\textwidth]{grafTTT.png}
\includegraphics[width=0.32\textwidth]{grafFTT.png}
\includegraphics[width=0.32\textwidth]{grafFFT.png}
\caption{Primeri grafov z $(i),(ii),(iii) =(\text{True},\text{True},\text{True})$, $(\text{False},\text{True},\text{True})$ in $(\text{False},\text{False},\text{True})$.}
\end{figure}



\section{Zaključek}
V nalogi smo preučili nekaj strukturnih lastnosti grafov. Izkazalo se je da pogojema $(i)$ in $(ii)$ 
ustreza zelo malo grafov, saj sta pogoja zelo stroga, za pogoj $(iii)$ je takšnih grafov več. 
Petersenov graf in grafi $K{r,r}$ za $r$ je liho število so posebni zaradi svoje strukture in ustrezajo vsem pogojem, 
hkrati pa spadajo v skupino močno regularnih grafov $\mathrm{srg}(n,r,\lambda = 0,\mu)$. Te grafe je zanimivo preučevati,
 saj jih veliko ustreza navedenim pogojem, predvsem pogoju $(iii)$.

\section{Priloga}
\label{sec:priloga}

\begin{figure}[H]
\centering
\includegraphics[width=0.48\textwidth]{i1.png}
\includegraphics[width=0.24\textwidth]{i2.png}
\includegraphics[width=0.24\textwidth]{i3.png}
\caption{Grafi, ki ustrezajo i). Po vrsti: Petersenov graf, $K_{3,3}$, $K_{5,5}$, $K_{7,7}$.}
\end{figure}


\begin{figure}[H]
\centering
\includegraphics{graph_r5_n16_idx379.png}
\caption{Graf, ki ustreza ii) za $n=16$, $r=5$}
\end{figure}

\begin{figure}[H]
\centering
\includegraphics{graph_r5_n16_idx383.png}
\caption{Graf, ki ustreza ii) za $n=16$, $r=35}
\end{figure}




\begin{figure}[H]
\centering
\includegraphics{graph_r7_n14_idx1.png}
\caption{Graf, ki ustreza ii) za $n=14$, $r=7$}
\end{figure}




\begin{figure}[H]
\centering
\includegraphics{n6r3.png}
\caption{Graf, ki ustreza iii) za $n=6$, $r=3$}
\end{figure}

\begin{figure}[H]
\centering
\includegraphics{n10r3.png}
\caption{Grafa, ki ustrezata iii) za $n=10$, $r=3$}
\end{figure}

\begin{figure}[H]
\centering
\includegraphics{n12r3.png}
\caption{Grafi, ki ustrezajo iii) za $n=12$, $r=3$}
\end{figure}

\begin{figure}[H]
\centering
\includegraphics{n14r3.png}
\caption{Grafi, ki ustrezajo iii) za $n=12$, $r=3$ (8 od 23)}
\end{figure}

\begin{figure}[H]
\centering
\includegraphics{n16r3.png}
\caption{Grafi, ki ustrezajo iii) za $n=16$, $r=3$ (8 od 13)}
\end{figure}

\begin{figure}[H]
\centering
\includegraphics{n10r5.png}
\caption{Graf, ki ustreza iii) za $n=10$, $r=5$}
\end{figure}

\begin{figure}[H]
\centering
\includegraphics{n16r5.png}
\caption{Grafi, ki ustrezajo iii) za $n=16$, $r=5$ (8 od 32)}
\end{figure}

\begin{figure}[H]
\centering
\includegraphics{random_walk_SOIS_n50_r3.png}
\caption{Časovni graf števila “SOIS-neighborhoods” med random walk}
\end{figure}




\end{document}